\section{Introduzione}
\label{introduzione}

\subsection{Abstract}
Il progetto scelto dal gruppo ha come scopo la realizzazione di un sito dedicato ad un professionista nel campo medico di nome Marco Donati. Esso si specializza nei disturbi auditivi, posturali e nasali.

L’obiettivo della pagina è quello di far conoscere le capacità del dottore e metterlo in contatto con i propri pazienti.
Il sito web offre quindi quattro funzionalità principali:
\begin{itemize}
\item La possibilità di creare un account privato col quale inviare messaggi privati tra l’utente e il dottore;
\item Una funzione di prenotazione visita online;
\item Le informazioni sulla locazione dello studio in cui saranno svolti i vari servizi medici disponibili e tutte le informazioni relative;
\item Una pagina nella quale verranno mostrate varie news, come eventuali chiusure dell’ufficio.
\end{itemize}

\subsection{Analisi d'utenza}
Il target del sito si concentra su uomini e donne di qualunque età in cerca di un consulto medico o di una prestazione medica specialistica.
Data l’ampiezza d’utenza potenzialmente molto alta e variegata, il numero di utenti non avvezzi alla navigazione web potrebbe essere elevato, quindi si è deciso di sviluppare un design semplice ed efficace.
Le informazioni sono quindi fornite tramite l’utilizzo di un linguaggio informale e comprensibile, restando però nei limiti di comprensione imposti dal campo medico.
Il sito infatti mantiene comunque la professionalità e precisione nei contenuti che si aspetterebbe un medico in visita nel sito.

Le informazioni fornite dal nostro sito sono di carattere puramente informativo, pensate per dare un’idea all’utente prima che incontri direttamente il dottore per spiegazioni molto più dettagliate.
Questa caratteristica fa sì che non sia necessaria una barra di ricerca, poiché la gran parte delle informazioni più complesse sarà data a voce o tramite il servizio di consulto online.
Si è pensato quindi di non doverla aggiungere in quanto l’espressività della navbar e delle sue sottosezioni copre completamente tutti i contenuti e le necessità dell’utente.

\subsection{Interazioni utente}

Abbiamo individuato 3 tipi di utente:
\begin{itemize}
\item Utente non registrato;
\item Utente registrato;
\item Admin.
\end{itemize}

L’utente non registrato può accedere a tutta la parte non interagibile del sito, ma può comunque informarsi su tutto il contenuto chiave senza doversi registrare.

L’utente registrato sblocca le funzionalità cardine del nostro servizio: la possibilità di prenotare un appuntamento tramite il nostro form online e consultare direttamente il Dottore in modo privato.
Inoltre dopo aver prenotato una visita l’utente può visualizzarne privatamente le informazioni a riguardo ,così da poterle anche stampare per documentarsi.
Questa funzionalità è valida anche per tutte le conversazioni private con il Dottore, queste vengono mostrate tramite chat a schermo e possono essere stampate.

L’admin del sito è lo stesso dottor Marco Donati, il quale ha poteri diversi dal semplice utente registrato, egli può:

\begin{itemize}
\item Aggiornare la pagina notizie, eliminando quelle vecchie e scrivendone di nuove;
\item Rispondere alle domande degli utenti registrati e visualizzare un elenco delle chat con tutti gli utenti;
\item Visualizzare una lista di tutte le prenotazioni delle visite, divise per giorno e tipologia.
\end{itemize}

Da notare il fatto che l’utente registrato e l’admin operano su due fronti diversi, l’admin non può modificare le prenotazioni fatte dall’utente se non in casi di emergenza dopo una notifica diretta paziente/dottore.