\section{Resoconto}
	\label{Resoconto}
	
	\subsection{Nome del gruppo}
	Dopo una preventiva riflessione individuale e una breve discussione generale dove ognuno ha indicato spunti o preferenze è stato scelto ProApes come nome ufficiale del gruppo.
	
	\subsection{Mail del gruppo}
	Il gruppo unanime nella decisione ha optato per proapes11@gmail.com come mail ufficiale del gruppo. Inoltre è stata considerata favorevole l'idea di sfruttare GoogleDrive come interfaccia per lo scambio di documenti utili al gruppo.	
	
	\subsection{Logo del gruppo}
	È stato approvato il logo, disegnato da Federico \footnote{Prova footnote} nei giorni scorsi, e si è deciso di
	mantenerne due versioni: una più compatta e grafica, adatta per esempio alle miniature, l'altra
	sviluppata più in lunghezza e quindi adatta alle intestazioni dei vari documenti e delle
	presentazioni.
	Si è discusso se mantenere il logo a colori o portarlo in bianco e nero, optando per il
	mantenimento di entrambe le colorazioni, da usare a seconda del contesto.
	
	\subsection{ Frequenza degli incontri}
	Dopo aver discusso delle disponibilità dei singoli membri, si è giunti alla conclusione che,
	almeno in queste prime settimane, il gruppo si riunirà con cadenza settimanale, probabilmente di
	martedì, per fare il punto della situazione, analizzare il procedere delle tappe iniziali del lavoro,
	risolvere insieme eventuali dubbi, accordarsi su questioni normative o altro.

	\subsection{Repository GitHub}
	È stato deciso di utilizzare Git come sistema di versionamento del progetto. Successivamente i membri del gruppo hanno valutato se sia opportuno adibire una repository remota solo
	per l'elaborazione e mantenimento dei documenti che saranno prodotti durante il progetto. Per
	andare incontro al principio di separazione, il gruppo ha deciso di separare lo sviluppo del codice
	da quello dei documenti, e si è quindi deciso di aprire la nuova repository, il cui link verrà
	opportunamente indicato nella repository principale del progetto.
	Questa seconda repository sarà creata su GitHub, come già la principale, anche se il
	gruppo si riserva di usare altre piattaforme di versionamento messi a disposizione in rete.

	\subsection{\LaTeX}
	Per strutturare in maniera professionale i documenti che saranno prodotti, il gruppo ha
	deciso di scrivere un template in linguaggio LaTex, che conterrà il frontespizio, gli indici,
	l'introduzione e i dettagli di impaginazione.

	\subsection{Analisi dei documenti per la RR}Il gruppo ha studiato le caratteristiche dello Studio di fattibilità, delle Norme di Progetto,
	dei Verbali (interni ed esterni) e della Lettera di presentazione. Dopo aver messo a fuoco gli
	elementi strutturali e contenutistici fondamentali di tali documenti, si è deciso di concentrarsi sullo
	Studio di fattibilità, poiché è parso quello che richiedesse una interazione attiva maggiore di tutti
	i membri; è stata prodotta una prima bozza del documento, che però non risulta ancora completo
	in tutte le sue parti.