\section{Implementazione lato back-end}
\subsection{JavaScript}
Per il comportamento di alcune pagine del sito, lato client, è stato utilizzato uno script JavaScript, comprendente diverse funzioni. È stato scelto di limitare al massimo l’utilizzo di queste funzionalità, in quanto non è possibile fare nessuna assunzione sulla disponibilità di questa tecnologia. Il sito infatti rimane completamente navigabile e utilizzabile nel caso in cui JavaScript venga disabilitato o non sia comunque disponibile. Sono state quindi fornite tutte le alternative, lato server, attraverso PHP (unica alternativa in questo senso è rappresentata dallo slideshow, che degrada elegantemente mostrando solo una immagine statica).
Da mobile si presuppone ovvimante JavaScript attivo. \\

Le principali funzioni che abbiamo implementato (tutte contenute all’interno di un unico file) sono state:
\begin{itemize}
\item \textbf{validaAccedi()} : questa funzione controlla che tutti i campi immessi nel form specifico per l'accesso al sito siano corretti;
\item \textbf{validaRegistratii()} : questa funzione controlla che tutti i campi immessi nel form specifico per la registrazione al sito siano corretti;
\item \textbf{setSubmitForJS()} : viene eseguita solo se c'è javascript; questa funzione fa si che, per chi abbia JavaScript attivo, vengano fatti i controlli sui campi prima di inviare il form, mentre per chi non dovesse avere JavaScript attivo, non venga eseguita questa funzione e venga lasciata la pagina così com'è, comunque utilizzabile;
\item \textbf{setPrenotaVisitaForJS()} : viene eseguita solo se c'è javascript;
questa funzione fa si che, per chi abbia JavaScript attivo, venga visualizzata la pagina come dovrebbe essere (con tutti i controlli e le chiamate alla pagina PHP per controllare la disponibilità), mentre per chi non dovesse avere JavaScript attivo, non venga eseguita questa funzione e quindi venga lasciata la pagina così com'è impostata in HTML, comunque utilizzabile;
\item \textbf{isValidDate(date)} : funzione che controlla che la data sia inserita nel giusto formato (d/m/y);
\item \textbf{controllaDisponibilita()} : funzione che nasconde gli orari se non si riferiscono più al giorno selezionato, in caso di cambio data da parte dell'utente;
\item \textbf{openCloseMenu(menu)} : funzione per chiudere e aprire i menù a tendina nella versione mobile del sito;
\item \textbf{plusDivs(n)} : funzione per far scorrere lo slideshow;
\item \textbf{showDivs(n)} : funzione per mostrare le diverse immagini dello slideshow;
\item \textbf{changeFocusAccedi(event, campo)}: funzione che, al premere del tasto invio una volta che è stata inserita la email nel form per accedere al sito, passa il focus sul campo password, senza inviare direttamente il form. \\


\subsection{PHP}


\end{itemize}