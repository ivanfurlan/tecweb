\section{Introduzione}
\label{introduzione}


\subsection{Abstract}
Il progetto scelto dal gruppo ha come scopo la realizzazione di un sito dedicato ad un medico professionista di nome Marco Donati. Essendo un otorino, si specializza nei disturbi auditivi, posturali e nasali.

L’obiettivo della pagina è quello di far conoscere le capacità del dottore e metterlo in contatto con i propri pazienti.
Il sito web offre quindi quattro funzionalità principali:
\begin{itemize}
\item La possibilità di creare un account privato con il quale inviare messaggi privati tra l’utente e il dottore;
\item Una funzione di prenotazione visita online;
\item Le informazioni sulla locazione dello studio in cui saranno svolti i vari servizi medici disponibili e tutte le informazioni relative ad essi;
\item Una pagina nella quale verranno mostrate varie news, di tipo organizzativo piuttosto che informativo o di carattere divulgativo.
\end{itemize}

\subsection{Analisi d'utenza}
Il target del sito si concentra su uomini e donne di qualunque età in cerca di un consulto medico o di una prestazione medica specialistica.
Data l’ampiezza d’utenza potenzialmente molto variegata, il numero di utenti non avvezzi alla navigazione web potrebbe essere elevato, quindi si è deciso di sviluppare un design molto semplice ed efficace.
Le informazioni sono quindi fornite tramite l’utilizzo di un linguaggio informale e comprensibile, restando però nei limiti di comprensione imposti dal campo medico.
Il sito infatti mantiene comunque la professionalità e la precisione nei contenuti che si aspetterebbe un medico in visita nel sito.

Le informazioni fornite dal nostro sito sono di carattere puramente informativo, pensate per dare un’idea all’utente prima che incontri direttamente il dottore per spiegazioni molto più dettagliate.
Questa caratteristica fa sì che non sia necessaria una barra di ricerca, poiché la gran parte delle informazioni più complesse sarà data a voce o tramite il servizio di consulti online.
Si è pensato quindi di non doverla aggiungere in quanto l’espressività della barra di navigazione e delle sue sottosezioni copre completamente tutti i contenuti e le necessità dell’utente.

\subsection{Interazioni utente}

Abbiamo individuato 3 tipi di utente:
\begin{itemize}
\item Utente non registrato;
\item Utente registrato;
\item Admin.
\end{itemize}

L’utente non registrato può accedere a tutta la parte non interagibile del sito, ma può comunque informarsi su tutto il contenuto chiave senza doversi registrare.

L’utente registrato sblocca le funzionalità cardine del nostro servizio: la possibilità di prenotare un appuntamento tramite il nostro form online e consultare direttamente il dottore in modo privato.
Inoltre dopo aver prenotato una visita, l’utente può visualizzarne privatamente le informazioni a riguardo, così da poterle anche stampare per documentarsi.
Questa funzionalità è valida anche per tutte le conversazioni private con il dottore, le quali vengono mostrate tramite chat sullo schermo e possono essere eventualmente stampate.

L’admin del sito è lo stesso dottor Marco Donati, il quale ha poteri diversi dal semplice utente registrato. Esso può:

\begin{itemize}
\item Aggiornare la pagina notizie, eliminare quelle vecchie e scriverne di nuove (precisiamo che l'admin all'interno del corpo della notizia può inserire dei tag HTML per poter formattare il testo come meglio crede, oltre che  a poter aggiungere link utili);
\item Rispondere alle domande degli utenti registrati e visualizzare un elenco delle chat con tutti gli utenti;
\item Visualizzare una lista di tutte le prenotazioni delle visite, divise per giorno e tipologia.
\end{itemize}


